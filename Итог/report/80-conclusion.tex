\Conclusion % заключение к отчёту

В результате выполнения курсовой работы была разработана программа с пользовательским интерфейсом, которая предоставляет функционал для моделирования броуновского движения частиц короновирусной инфекции в помещении с учетом скорости их распространения и времени жизни на разных поверхностях.
Были решены следующие задачи:
\begin{itemize}
	\item рассмотрены алгоритмы удаления невидимых линий и поверхностей и методы закраски;
	\item проанализированы алгоритмы моделирования броуновского движения;
	\item выбраны и реализованы подходящие для решения поставленной задачи алгоритмы;
	\item формализована модель, представлена диаграмма классов;
	\item выявлена зависимость времени отрисовки кадра от количества частиц вируса, находящихся на сцене.
\end{itemize}

В ходе замеров было выяснено, что скорость отрисовки сцены линейно зависит от количества частиц вируса.