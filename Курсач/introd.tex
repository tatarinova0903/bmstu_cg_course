\chapter*{Введение}
\addcontentsline{toc}{chapter}{Введение}

 С развитием компьютерных технологий компьютерная графика приобрела совершенно новый статус, поэтому сегодня она является основной технологией в цифровой фотографии, кино, видеоиграх, а также во многих специализированных приложениях. Было разработано большое количество алгоритмов отображения. Главными критериями, которые к ним предъявляются, являются реалистичность изображения и скорость отрисовки. Однако зачастую чем выше реалистичность, тем больше времени и памяти требуется для работы алгоритма.

Одной из тем моделирования является моделирование движения частиц. Имеется огромная потребность в качественной и эффективной отрисовке распространения частиц вируса. Особенно эта тема стала актуальной после начала пандемии короновируса. Пандемия COVID-19 повлияла на жизнь миллионов людей по всему миру. Помимо серьезных последствий для здоровья, пандемия также изменила нашу повседневную жизнь, перевернула рынок вакансий и подорвала экономическую стабильность. В данном курсовом проекте речь пойдет о моделировании распростарнения частиц вирусной инфекции.

\textbf{Цель работы} -- разработать программное обеспечение для моделирования распрстранения частиц короновирусной инфекции в помещении:

\begin{itemize}
	\item проанализировать методы и алгоритмы, моделирующие броуновское движение частиц; 
	\item определить алгоритм, который наиболее эффективно справляется с поставленной задачей;
	\item реализовать алгоритм;
	\item разработать структуру классов проекта;
	\item провести эксперимент по замеру производительности полученного программного обеспечения.
\end{itemize}
