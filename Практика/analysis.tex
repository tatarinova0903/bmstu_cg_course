\chapter{Аналитическая часть}

\section{Анализ алгоритмов удаления невидимых линий и поверхностей}

\subsection{Алгоритм обратной трассировки}

Алгоритм обратной трассировки лучей отслеживает лучи в обратном направлении (от наблюдателя к объекту). Считается, что наблюдатель расположен на положительной полуоси z в бесконечности, поэтому все световые лучи параллельны оси z. В ходе работы испускаются лучи от наблюдателя и ищутся пересечения луча и всех объектов сцены.
В результате, пересечение с максимальным значением z является видимой частью поверхности, и атрибуты данного объекта используются для определения характеристик пикселя, через центр которого проходит данный световой луч. 

Эффективность процедуры определения пересечений луча с поверхностью объекта оказывает самое большое влияние на эффективность всего алгоритма. Чтобы избавиться от ненужного поиска пересечений было придумано искать пересечение луча с объемной оболочкой рассматриваемого объекта. Под оболочкой понимается некоторый простой объект, внутрь которого можно поместить рассматриваемый объект, к примеру параллелепипед или сферу. 

В дальнейшем при рассмотрении пересечения луча и объемной оболочкой рассматриваемого объекта, если такого пересечения нет, то и соответственно пересечения луча и самого рассматриваемого объекта нет, и наоборот, пересечение найдено, то возможно, есть пересечение луча и рассматриваемого объекта. 

Для расчета эффектов освещения сцены проводятся вторичные лучи от точек пересечения ко всем источникам света. Если на пути этих лучей встречается непрозрачное тело, значит данная точка находится в тени, иначе он влияет на освещение данной точки. Также для получения более реалистичного изображения сцены, нужно учитывать вклады отраженных и преломленных лучей.

Плюсы:

\begin{itemize}
	\item возможность использования алгоритма в параллельных вычислительных системах.
\end{itemize}

Минусы:

\begin{itemize}
	\item требуется большое количество высислений;
	\item производительность алгоритма.
\end{itemize}

\subsection{Алгоритм Робертса}

Алгоритм Робертса работает в объектном пространстве, кроме того работает только с выпуклыми телами. Если тело изначально является не выпуклым, то нужно его разбить на выпуклые составляющие.

Данный алгоритм состоит из следующих основных этапов:

\begin{itemize}
	\item подготовка исходных данных;
	\item удаление линий, экранируемых самим телом;
	\item удаление линий, экранируемых другими телами.
\end{itemize}

Плюсы:

\begin{itemize}
	\item высокая точность вычислений.
\end{itemize}

Минусы:

\begin{itemize}
	\item рост числа трудоемкости алгоритма, как квадрата числа объектов;
	\item работа только с выпуклыми телами.
\end{itemize}

\subsection{Алгоритм, использующий Z-буфер}

Алгоритм Z-буфера решает задачу в пространстве изображений. 

В данном алгоритме рассматривается два буфера. Буфер кадра (регенерации) используется для заполнения атрибутов (интенсивности) каждого пикселя в пространстве изображения. В Z-буфер (буфер глубины) можно помещать информацию о координате z для каждого пикселя.

Для начала необходимо подготовить буферы. Для этого в Z-буфер заносятся максимально возможные значения z, а буфер кадра заполняется значениями пикселя, который описывает фон. Также нужно каждый многоугольник преобразовать в растровую форму и записать в буфер кадра. Сам процесс работы заключается в сравнении глубины каждого нового пикселя, который нужно занести в буфер кадра,
с глубиной того пикселя, который уже занесён в Z-буфер. В зависимости от сравнения принимается решение, нужно ли заносить новый пиксель в буфер кадра и, если нужно, также корректируется Z-буфер (в него нужно занести глубину нового пикселя).

Плюсы:
\begin{itemize}
	\item элементы сцены заносятся в буфер кадра в произвольном порядке, поэтому в данном алгоритме не тратится время на выполнение сортировок;
	\item произвольная сложность сцены;
	\item поскольку размеры изображения ограничены размером экрана дисплея, трудоемкость алгоритма зависит линейно от числа рассматриваемых поверхностей.
\end{itemize}

Минусы:
\begin{itemize}
	\item трудоемкость устранения лестничного эффекта;
	\item трудности реализации эффектов прозрачности;
	\item большой объем требуемой памяти.
\end{itemize}

\subsection*{Вывод}

Для удаления невидимых линий и поверхностей выбран алгоритм Z-буфера, так как обладает важными преимуществами - высокой скоростью работы и произвольной сложностью сцены.


\section{Анализ методов закрашивания}

Методы закрашивания используются для затенения полигонов (или по-
верхностей, аппроксимированных полигонами) в условиях некоторой сцены, имеющей источники освещения.

Существует несколько основных методов закраски:
\begin{itemize}
	\item простая закраска;
	\item закраска по Гуро, основанная на интерполяции значений интенсивности освещенности поверхности;
	\item закраска по Фонгу, основанная на интерполяции векторов нормалей к граням многогранника.
\end{itemize}

\subsection{Простая закраска}

Одной из самых простых моделей освещения является модель Ламберта. Она учитывает только идеальное диффузное отражение света от тела. Считается, что свет падающий в точку, одинаково рассеивается по всем направлениям полупространства. Таким образом, освещенность в точке определяется только плотностью света в точке поверхности, а она линейно зависит от косинуса угла падения. При этом положение наблюдателя не имеет значение, т.к. диффузно отраженный свет рассеивается равномерно по всем направлениям.

Большим недостатком данной модели является то, что согласно приведённой выше формуле, все точки грани будут иметь одинаковую интенсивность.

\imgHeight{60mm}{draw_simple.png}{Пример простой закраски}

\subsection{Закраска по Гуро}

Данный алгоритм предполагает следующие шаги:

\begin{itemize}
	\item вычисление векторов нормалей к каждой грани;
	\item вычисление векторов нормали к каждой вершине грани путем усреднения нормалей к граням;
	\item вычисление интенсивности в вершинах грани;
	\item интерполяция интенсивности вдоль ребер грани;
	\item линейная интерполяция интенсивности вдоль сканирующей строки.
\end{itemize}

Плюсы:

\begin{itemize}
	\item хорошо сочетается с диффузным отражением;
	\item изображение получается более реалистичным, чем при простой закраске.
\end{itemize}

Минусы:

\begin{itemize}
	\item данный метод интерполяции обеспечивает лишь непрерывность значений интенсивности вдоль границ многоугольников, но не обеспечивает непрерывность изменения интенсивности.
\end{itemize}

\imgHeight{60mm}{draw_guro.png}{Пример закраски по Гуро}

\subsection{Закраска по Фонгу}

При такой закраске, в отличие от метода Гуро, вдоль сканирующей строки интерполируется значение вектора нормали, а не интенсивности. 

Шаги алгоритма:

\begin{itemize}
	\item вычисление векторов нормалей в каждой грани.
	\item вычисление векторов нормали к каждой вершине грани.
	\item интерполяция векторов нормалей вдоль ребер грани.
	\item линейная интерполяция векторов нормалей вдоль сканирующей строки.
	\item вычисление интенсивности в очередной точке сканирующей строки.
\end{itemize}

Плюсы:

\begin{itemize}
	\item можно достичь лучшей локальной аппроксимации кривизны поверхности.
\end{itemize}

Минусы:

\begin{itemize}
	\item ресурсоемкость;
	\item вычислительная сложность.
\end{itemize}

\imgHeight{60mm}{draw_phong.png}{Пример закраски по Фонгу}

\subsection*{Вывод}

Для закрашивания выбран алгоритм Фонга, так как данный алгоритм обладает важным преимуществом - высокой реалистичностью изображения.


\section*{Вывод}

В данном разделе были формально описаны алгоритмы удаления невидимых линий и поверхностей, методы закрашивания поверхностей. В качестве алгоритма удаления невидимых линий и поверхностей был выбран алгоритм Z-буфера, в качестве метода закрашивания был выбран алгоритм закраски Фонга.

