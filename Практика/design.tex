\chapter{Конструкторская часть}

\section{Понятие броуновского движения}

\textbf{Броуновское движение} (иногда называют Брауновское движение) -- беспорядочное движение малых частиц, взвешенных в жидкости или газе, происходящее под действием молекул окружающей среды. 

\imgHeight{60mm}{brown_movement.png}{Броуновское движение}

Броуновское движение представляет собой пример естественного фрактала с фрактальной размерностью $d$ = 1.5 (Мандельброт и Ван Несс, 1968). Впервые его наблюдал шотландский ботаник Роберт  Броун в 1827 году: он  заметил  непрерывное  беспорядочное движение взвешенных в жидкости маленьких частиц (пыльцы), но ошибочно приписал причину движения самим частицам. Только в 1905 году Альберт Эйнштейн и вслед за ним в 1906 году Мариан Смолуховский объяснили это движение хаотическими соударениями с молекулами окружающей среды. В 1908–1913 годах Жан Батист Перрен поставил ряд опытов, подтвердивших выводы Эйнштейна и Смолуховского. И, наконец, в 1923 год Норберт Винер построил первую математическую модель броуновского движения. Альтернативные подходы были предложены Андреем Николаевичем Колмогоровым в 1933 году и Полем Леви в 1948 году.

\section{Моделирование броуновского движения}

\subsection{Классическое броуновское движение}

Рассмотрим случайный процесс (случайную величину) \textit{X(t)}, заданную на отрезке [0,\textit{T}].

\textit{Случайный процесс}  \textit{X(t)} называется одномерным броуновским движением (или винеровским процессом) на интервале [0,\textit{T}], если он обладает следущими свойствами:
\begin{itemize} 
	\item \textit{X}(0) = 0 почти наверное и \textit{X(t)} - почти наверное непрерывная функция на [0,\textit{T}]
	\item \textit{X(t)} - процесс с независимыми приращениями
	\item  \textit{X(t)} -  процесс с приращениями, распределёнными нормально.
\end{itemize}

Отметим следующие свойства броуновского движения:
\begin{itemize} 
	\item \textit{X(t)} почти наверное нигде не дифференцируем 
	\item  \textit{X(t)} - марковский процесс (не обладает памятью), т.е. если известна величина \textit{X(t)}, то при $t_1$ < $t$ < $t_2$ величины \textit{X($t_1$)} и \textit{X($t_2$)} независимы.
	\item Фрактальная размерность графика \textit{X(t)} равна 1.5
	\item Приращение \textit{X(t)} обладает свойством статистического самоподобия: для любого $r$ > 0
	\begin{equation}
		X(t+ \bigtriangleup t) = \frac{1}{\sqrt{r}}(X(t+r\bigtriangleup t) - X(t))
	\end{equation}
	\item Стационарность приращений: дисперсия приращения зависит только от разности моментов времени
	\begin{equation} \label{1.6}
		D(X(t_2) - X(t_1)) = \sigma^2|t_2-t_1|
	\end{equation}
	\item Математическое ожидание приращения равно
	\begin{equation}
		E(|X(t_2) - X(t_1)|) = \sqrt{\frac{2}{\pi}}\sigma\sqrt{|t_2-t_1|}
	\end{equation}
\end{itemize}

Для моделрования броуновского движения можно воспользоваться разными алгоритмами. Рассмотрим 3 из них.

Проще всего реализовать дискретную реализацию броуновского движения, рассмотрев последовательность $x_0$ = 0, $x_{n+1} = x_n + g_n$, где $g_n$ - случайная величина, имеющая нормальное распределение (например, $N(0,1)$).

\begin{algorithmic}[1]
	\State $array[N]$
	\State $array[0]\gets 0$
	\For{i = 1,..., $N$}
	\State $array[i+1]\gets array[i] + randomNormal(0,1)$
	\EndFor
\end{algorithmic}

\subsection{Алгоритм срединных смещений}

Метод случайного срединного смещения основан на работах Н.Виннера , он более сложен, чем метод из предыдущего параграфа, однако используется для конструктивного доказательства существования броуновского движения, а также для построения фрактальной интерполяции (когда необходимо чтобы кривая проходила через заданные точки интерполяции). Метод также может быть обобщен на случай $n$-мерных броуновских движений.

Алгоритм случайного срединного смещения вычисляет значения $X(t)$ в диадических рациональных точках вида $\frac{k}{2^n}$ $\in$ [0,1]. Последовательно вычисляются значения в середине отрезка [0,1], а затем в серединах отрезков [0, $\frac{1}{2}$] и [$\frac{1}{2}, 1$] и т.д. На каждом шаге итерации должен выполнятьяс закон дисперсии для приращения в вычисленных точках. Параметр $\sigma$ определяет масщтаб по вертикальной оси, не влияя на фрактальную размерность графика. 

\subsubsection{Броуновское движение методом срединнго смещения}

Вход: $N$, 	$\sigma$ // $N$ - число шагов алгоритма, при этом всего $2^N + 1$ точек интерполяции, $\sigma$ - параметр вертикального масштаба, коэффициент дисперсии

Выход: массив значений $\left\{X(\frac{k}{2^N})\right\}_{k=0}^{2^N}$ // реализация броуновского движения $X(t)$ на дискретном множестве точек вида $t_k = \frac{k}{2^N}$, k $\in$ $[0, 2^N]$

\begin{algorithmic}[1]
	\State $X(0)\gets 0$
	\State $X(1)\gets \sigma g$ // g - случайная величина, распределенная нормально с параметрами $N(0,1)$
	\For{j = 1,..., N}
	\For{i = 1,..., $2^{N-1}$}
	\State $X((2i - 1) 2^{N - j})$$\gets$ $X((i - 1)2^{N-j+1}) + X(i2^{N - j + 1}) + \frac{1}{2^{(j+1)/2}}$$\sigma$ g
	\EndFor
	\EndFor
\end{algorithmic}

\subsection{Фрактальное броуновское движение}

Фрактальное броуновское движение (ФБД) уже не является марковским процессом, а обладает некорой "памятью". Кроме того, вводя параметр $0 < H < 1$ можно получитьодномерное ФБД размерности $d = 2 - H$ и двумерное ФБД размерности $d = 3 - H$ .
Заметим, что классическое броуновское движение получается как частный случай при $H = 0.5$. Для апроксимации ФБД нет простого метода, вроде суммирования нормальных случайных величин, как в случае классического броуновского движения. Для апроксимации ФБД наиболее удобно использовать преобразования Фурье.

Рассмотрим случайный процесс (случайную величину) $X(t)$, заданную на отрезке $[0, T]$.

\textit{Случайный процесс}  $X(t)$ называется одномерным фрактальным броуновским движением на интервале $[0, T]$, если он обладает следущими свойствами:

\begin{itemize}
	\item $X(0) = 0$ почти наверное и $X(t)$ - почти наверне непрерывная функция на $[0, T]$
	\item  $X(t)$ - процесс с приращениями, распределенными нормально
\end{itemize}

Отметим следующие свойства фрактального броуновского движения:

\begin{itemize}
	\item $X(t)$ почти наверное нигде не дифференцируем 
	\item Фрактальная размерность графика $X(t)$ равна $2 - H$
	\item Процесс $x(t)$ не обладает свойством независимости приращений
	\item Приращение \textit{X(t)} обладает свойством статистического самоподобия: для любого $r$ > 0
	\begin{equation}
		X(t+ \bigtriangleup t) = \frac{1}{\sqrt{r}}(X(t+r\bigtriangleup t) - X(t))
	\end{equation}
	\item Стационарность приращений: дисперсия приращения зависит только от разности моментов времени
	\begin{equation} \label{1.6}
		D(X(t_2) - X(t_1)) = \sigma^2|t_2-t_1|^{2H}
	\end{equation}
	\item Математическое ожидание приращения равно
	\begin{equation}
		E(|X(t_2) - X(t_1)|) = \sqrt{\frac{2}{\pi}}\sigma|t_2-t_1|^H
	\end{equation}
\end{itemize}

\subsubsection{Метод Фурье-фильтрации для построения ФБД}

\newtheorem{theorem1}{Теорема}
\begin{theorem1}
	Если $X(t)$ - ФБД с параметром $H$, то его спектральная плотность 
	\begin{equation}
		S(f) \propto \frac{1}{f^{2H+1}}
	\end{equation}
\end{theorem1}

Идея метода состоит в следующем. Строится преобразование Фурье для искомого ФБД в частной области, задавая случайные фазы и подбирая амплитуды, удовлетворяющие свойству из Теоремы 1. Затем получаем ФБД во временной области с помощью обратного преобразования Фурье.

Будем моделировать дискретный аналог ФБД, то есть наша цель- получить величины $\left\{X_n\right\}_{n=0}^{N-1}$, апроксимирующиеФБД в точках $n$. Воспользуемся формулой дискретного преобразования Фурье

\begin{equation}
	\hat{X_n} = \sum_{k=0}^{N-1}X_ke^{-2\pi kn/N}
\end{equation}

и обратного дискретного преобразования Фурье

\begin{equation}
	X_n = \sum_{k=0}^{N-1}\hat{X_k}e^{2\pi kn/N}
\end{equation}

Далее будем рассматривать только четные значения $N$, а для применения метода \textit{быстрого дискретного преобразования Фурье} нужно, чтобы $N=2^M$, $M \in \mathbb{N}$. Метод быстрого дискретного преобразования Фурье  реализован во многих системах компьютерной алгебры. Он позволяет сократить вычисления в $\frac{2N}{\log_2N}$ раз.

Для того, чтобы получающиеся величины $X_n$ были вещественными мы наложим условие сопряженной симметрии:

\begin{equation}
	\hat{X}_0, \hat{X}_{N/2} \in \mathbb{R}, \hat{X}_n = \hat{X}_{N-n}, n = 1,...,N/2-1
\end{equation}

Фильтрация относится к той части моделирования, когда мы заставляем коэффициенты преобразования Фурье удовлетворять степенному закону из Теоремы 1:

\begin{equation}
	|\hat{X}_n|^2 \propto \frac{1}{n^{2H+1}}, n = 1,...,N/2
\end{equation}

Для этого возьмем

\begin{equation}
	\hat{X}_n = \frac{ge^{2\pi iu}}{n^{H+0.5}}
\end{equation}

где $g$ - независимые значения нормально распределенной случайной величины с параметрами $N(0,1)$, а $u$ - независимые значения равномерно распределенной на отрезке [0,1] случайной величины. Оставшиеся коэффициенты вечислим из сотношений 1.15.

Для вычисления искомой аппроксимации ФБД $\left\{X_n\right\}_{n=0}^{N-1}$ применим обратное дискретное преобразование Фурье к набору $\left\{\hat{X}_n\right\}_{n=0}^{N-1}$.

\subsubsection{Кривая ФБД методом Фурье-фильтрации}

Вход: $H \in (0,1)$, $N=2^M$, $M \in \mathbb{N}$ // $H$ - параметр ФБД, размерность графика равна $d = 2 - H$, $N$ - параметр, определяющий количество точек дискретизации ФБД.

Выход: массив значений $\left\{X_n\right\}_{n=0}^{N-1}$ // дискретная апроксимация ФБД в последовательные моменты времени n.

\begin{algorithmic}[1]
	\State $\hat{X}_0\gets g$
	\For{j = 1,..., N/2-1}
	\State $\hat{X}_j \gets \frac{ge^{2\pi iu}}{j^{H+0.5}}$
	\EndFor
	\State $\hat{X}_{N/2} \gets \frac{g\cos(2\pi iu)}{(N/2)^{H+0.5}}$ // Здесь $\cos$ — вещественная часть комплексной экспоненты $e$
	\For{j = N/2+1,..., N-1}
	\State $\hat{X}_j \gets \overline{\hat{X}_{N-j}}$
	\EndFor
	\State $X \gets convert(\hat{X})$ // Вектор $X = \left\{X_0,...,X_{N-1}\right\}$ получается обратным дискретным преобразованием Фурье из вектора $\hat{X} = \left\{\hat{X}_0,...,\hat{X}_{N-1}\right\}$.
\end{algorithmic}

Для построения апроксимации двумерного фрактального броуновского движения методом Фурье-фильтрации используются те же идеи, что и в одномерном случае. Вместо $\hat{X}_n$ используется $\hat{X}_{k,j}$, $k,j = \overline{0,N-1}$, условие Теоремы 1 примет вид:

\begin{equation}
	|\hat{X}_{k,j}|^2 \propto \frac{1}{(k^2+j^2)^{H+1}}, n,k= 1,...,N/2
\end{equation}

мы возьмем

\begin{equation}
	\hat{X}_{k,j} = \frac{ge^{2\pi iu}}{(k^2+j^2)^{H/2+0.5}}, n,k= 1,...,N/2
\end{equation}

Запишем обратное дискретное преобразование Фурье: для $m,n=\overline{0,N-1}$

\begin{gather}
	\begin{split}
		\hat{X}_{m,n} = \sum_{k=0}^{N-1}\sum_{j=0}^{N-1}\hat{X}_{k,j}e^{-2\pi i \frac{kn+jm}{N}} = \hat{X}_{0,0} + \sum_{k=1}^{N-1}\hat{X}_{k,0}e^{-2\pi i\frac{kn}{N}} + \sum_{j+1}^{N-1}\hat{X}_{0,j}e^{-2\pi i \frac{jm}{N}} + \\
		\sum_{k=1}^{N/2}\sum_{j=1}^{N/2}\hat{X}_{k,j}e^{-2\pi i \frac{kn+jm}{N}} + \sum_{k=\frac{N}{2}+1}^{N-1} \sum_{j=\frac{N}{2}+1}^{N-1}(...) + \sum_{k=1}^{N/2}\sum_{j=\frac{N}{2}+1}^{N-1}(...) + \sum_{k=\frac{N}{2}+1}^{N-1} \sum_{j=1}^{N/2}(...)
	\end{split}
\end{gather}

Из формулы (1.19) следует, что для вещественности всех величин $X_{m,n}$ достаточно выполнения следующих условий сопряженной симметрии:

\begin{align}
	\hat{X}_{N-k,N-j} &= \overline{\hat{X}_{k,j}} &k,j &= \overline{1,N/2} &\hat{X}_{N/2,N/2} & \in \mathbb{R}\\
	\hat{X}_{k,N-j} &= \overline{\hat{X}_{N-k,j}} &k,j &= \overline{1,N/2-1} &\hat{X}_{0,0} & \in \mathbb{R}\\
	\hat{X}_{0,N-j} &= \overline{\hat{X}_{0,j}} &j &= \overline{1,N/2} &\hat{X}_{0,N/2} & \in \mathbb{R}\\
	\hat{X}_{N-k,0} &= \overline{\hat{X}_{k,0}} &k &= \overline{1,N/2} &\hat{X}_{N/2,0} & \in \mathbb{R}
\end{align}

Условия (1.22)-(1.23) обеспечивают вещественность первых двух сумм, а условия (1.20)-(1.21) - оставшихся четырех сумм.

\subsection*{Вывод}

Алгоритм Фурье-фильтрации содержит большое количество сложных вычислений, которые отрицательно влияют на скорость работы программы. Однако он позволяет наиболее реалистично изобразить броуновское движение, а также легко обобщается для случая $n$-мерных движений. Поэтому для реализации поставленной задачи был выбран именно этот алгоритм.

\section{Формализация модели}
Модель броуновского движения частиц будет задаваться такими характеристиками, как:
\begin{itemize} 
	\item скорость распространения -- число типа $int$;
	\item количество частиц -- число типа $int$.
\end{itemize}

Также на сцене будет изображено пустое помещение и абстрактная фигура человека в его центре. Пользователь должен уметь задавать материал покрытия стен:
\begin{itemize} 
	\item дерево;
	\item бумага (обои).
\end{itemize}
и пола:
\begin{itemize} 
	\item дерево (паркет);
	\item керамика (плитка).
\end{itemize}

\section{Требования к программному обеспечению}

Программа должна предоставлять доступ к функционалу:

\begin{itemize}
    \item возможность выбора материала покрытия пола и стен из предложенных вариантов (дерево, бумага(обои), керамика(плитка));
    \item изменение скорости движения;
    \item изменение количества частиц инфекции;
    \item включение и выключение работы модели распространения частиц;
    \item вращение, перемещение и масштабирование модели.
\end{itemize}

Требования, которые предъявляются к программе:

\begin{itemize}
    \item время отклика программы должно быть менее 1 секунды для корректной работы в интерактивном режиме;
    \item программа должна корректно реагировать на любое действие пользователя.
\end{itemize}


\section*{Вывод}

В данном разделе были рассмотрены основные алгоритмы для реализации поставленной задачи, т.е. моделирования броуновскго движения частиц. В качестве основного алгоритма был выбран метод Фурье-фильтрации. Также была формализована модель и определены требования, которые выдвигаются к программному продукту. 
