\chapter{Конструкторская часть}

В данном разделе будут представлены требования к программному обеспечению, а также схемы алгоритмов, выбранных для решения задачи.

\section{Формализация модели}
Модель броуновского движения частиц будет задаваться такими характеристиками, как:
\begin{itemize} 
	\item скорость распространения -- маляа, средняя или высокая;
	\item количество частиц -- число типа $int$.
\end{itemize}

Также частью на сцене будет изображено помещение. Пользователь должен уметь задавать материал покрытия стен:
\begin{itemize} 
	\item дерево;
	\item бумага (обои).
\end{itemize}
и пола:
\begin{itemize} 
	\item дерево (паркет);
	\item керамика (плитка).
\end{itemize}

\section{Выбор программного обеспечения}

Рендеринг или отрисовка -- термин в компьютерной графике, обозначающий процесс получения изображения по модели с помощью компьютерной программы.

В языке Java есть несколько основных инструментов для создания пользовательских изображений. Самыми популярными из них являются JavaFX и Swing.

\subsection{Swing}

\textbf{Swing} -- библиотека для создания графического интерфейса для программ на языке Java. Swing был разработан компанией SunMicrosystems. Он содержит ряд графических компонентов, таких как кнопки, поля ввода, таблицы и т. д.

\textbf{Преимущества:} 
\begin{itemize}
	\item Кроссплатформенность;
	\item Компоненты Swing следуют парадигме Model-View-Controller (MVC) и, таким образом, могут обеспечить гораздо более гибкий пользовательский интерфейс;
	\item Swing обеспечивает встроенную двойную буферизацию.
\end{itemize}

\textbf{Недостатки:} 
\begin{itemize}
	\item достаточно узкий спектр возможностей при работе с ui.
	\item считается устаревшей библиотекой.
\end{itemize}


\subsection{JavaFX}

\textbf{JavaFX} -- платформа на основе Java для создания приложений с насыщенным графическим интерфейсом. Может использоваться как для создания настольных приложений, запускаемых непосредственно из-под операционных систем, так и для интернет-приложений, работающих в браузерах, и для приложений на мобильных устройствах. 

JavaFX предназначен для предоставления приложениям таких сложных функций графического интерфейса, как плавная анимация, просмотр веб-страниц, воспроизведение аудио и видео, а также использование CSS стилей.

\textbf{Преимущества:} 
\begin{itemize}
	\item кроссплатформенность;
	\item больше встроенных возможностей;
	\item меньше кода.
\end{itemize}

\textbf{Недостатки:} 
\begin{itemize}
	\item технология еще молодая и "незрелая";
	\item в значительной степени зависит от огромной инфраструктуры, которая окружает Java.
\end{itemize}



\subsection*{Вывод}

Уже более 10 лет разработчики приложений считают Swing высокоэффективным инструментарием для создания графических пользовательских интерфейсов (GUI) и добавления интерактивности в Java-приложения. Однако некоторые из самых популярных на сегодняшний день функций графического интерфейса не могут быть легко реализованы с помощью Swing в отличии от JavaFX. 

Также можно писать программы на JavaFX, используя гораздо меньше кода, потому что JavaFX выполняет за нас всю работу. Не нужно регистрировать event listeners, и это делает тело функций более кратким. Кроме того, с помощью механизма привязки JavaFX легко интегрировать компоненты графического интерфейса с базовой моделью. Основываясь на вышесказанном в качестве библиотеки для работы с GUI была выбрана JavaFX.


\section{Требования к программному обеспечению}

Программа должна предоставлять доступ к функционалу:

\begin{itemize}
    \item возможность выбора материала покрытия пола и стен из предложенных вариантов (дерево, бумага(обои), керамика(плитка));
    \item изменение скорости движения;
    \item изменение количества частиц инфекции;
    \item включение и выключение работы модели распространения частиц;
    \item вращение, перемещение и масштабирование модели.
\end{itemize}

Требования, которые предъявляются к программе:

\begin{itemize}
    \item время отклика программы должно быть менее 1 секунды для корректной работы в интерактивном режиме;
    \item программа должна корректно реагировать на любое действие пользователя.
\end{itemize}


\section*{Вывод}

В данном разделе были рассмотрены требования, которые выдвигаются программному продукту, а также определеы технологии, которые будут использованы при реализации ПО. 
