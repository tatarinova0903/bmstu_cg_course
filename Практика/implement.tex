\chapter{Технологическая часть}

В данном разделе будут рассмотрены средства разработки программного обеспечения, а также детали реализации.


\section{Средства реализации}

При написании программного продукта был выбран язык \textit{Java}. Это обусловлено следующими факторами:

\begin{itemize}
    \item объектно-ориентированный язык, что позволяет использовать структуру классов;
    \item имеются необходимые библиотеки для реализации поставленной задачи;
    \item существует много учебной литературы.
\end{itemize}

В качестве среды разработки был выбран \textit{IntelliJ IDEA}. \textit{IntelliJ IDEA} -- это интеллектуальная IDE, учитывающая контекст. Она предназначена для разработки разнообразных приложений на Java и других языках JVM, например Kotlin, Scala и Groovy. Также она поддерживает Git.

\section{Выбор программного обеспечения}

Рендеринг или отрисовка -- термин в компьютерной графике, обозначающий процесс получения изображения по модели с помощью компьютерной программы.

В языке Java есть несколько основных инструментов для создания пользовательских изображений. Самыми популярными из них являются JavaFX и Swing.

\subsection{Swing}

\textbf{Swing} -- библиотека для создания графического интерфейса для программ на языке Java. Swing был разработан компанией SunMicrosystems. Он содержит ряд графических компонентов, таких как кнопки, поля ввода, таблицы и т. д.

\textbf{Преимущества:} 
\begin{itemize}
	\item Кроссплатформенность;
	\item Компоненты Swing следуют парадигме Model-View-Controller (MVC) и, таким образом, могут обеспечить гораздо более гибкий пользовательский интерфейс;
	\item Swing обеспечивает встроенную двойную буферизацию.
\end{itemize}

\textbf{Недостатки:} 
\begin{itemize}
	\item достаточно узкий спектр возможностей при работе с ui.
	\item считается устаревшей библиотекой.
\end{itemize}


\subsection{JavaFX}

\textbf{JavaFX} -- платформа на основе Java для создания приложений с насыщенным графическим интерфейсом. Может использоваться как для создания настольных приложений, запускаемых непосредственно из-под операционных систем, так и для интернет-приложений, работающих в браузерах, и для приложений на мобильных устройствах. 

JavaFX предназначен для предоставления приложениям таких сложных функций графического интерфейса, как плавная анимация, просмотр веб-страниц, воспроизведение аудио и видео, а также использование CSS стилей.

\textbf{Преимущества:} 
\begin{itemize}
	\item кроссплатформенность;
	\item больше встроенных возможностей;
	\item меньше кода.
\end{itemize}

\textbf{Недостатки:} 
\begin{itemize}
	\item технология еще молодая и "незрелая";
	\item в значительной степени зависит от огромной инфраструктуры, которая окружает Java.
\end{itemize}



\subsection*{Вывод}

Уже более 10 лет разработчики приложений считают Swing высокоэффективным инструментарием для создания графических пользовательских интерфейсов (GUI) и добавления интерактивности в Java-приложения. Однако некоторые из самых популярных на сегодняшний день функций графического интерфейса не могут быть легко реализованы с помощью Swing в отличии от JavaFX. 

Также можно писать программы на JavaFX, используя гораздо меньше кода, потому что JavaFX выполняет за нас всю работу. Не нужно регистрировать event listeners, и это делает тело функций более кратким. Кроме того, с помощью механизма привязки JavaFX легко интегрировать компоненты графического интерфейса с базовой моделью. Основываясь на вышесказанном в качестве библиотеки для работы с GUI была выбрана JavaFX.


\section*{Вывод}

В данном разделе были рассмотрены средства, с помощью которых было решено реализовывать ПО, технологии, которые будут использованы при его реализации.
