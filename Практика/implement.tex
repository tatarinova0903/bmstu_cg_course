\chapter{Технологическая часть}

\section{Средства реализации}

При написании программного продукта был выбран язык \textit{Java}. Это обусловлено следующими факторами:

\begin{itemize}
	\item объектно-ориентированный язык, что позволяет использовать структуру классов;
	\item имеются необходимые библиотеки для реализации поставленной задачи;
	\item существует много учебной литературы.
\end{itemize}

В качестве среды разработки была выбрана \textit{IntelliJ IDEA}. \textit{IntelliJ IDEA} -- это интеллектуальная IDE, учитывающая контекст. Она предназначена для разработки разнообразных приложений на Java и других языках JVM, например Kotlin, Scala и Groovy. Также она поддерживает Git.

\section{Выбор программного обеспечения}

В языке Java есть несколько основных инструментов для создания пользовательских изображений. Самыми популярными из них являются JavaFX и Swing.

\subsection{Swing}

\textbf{Swing} -- библиотека для создания графического интерфейса для программ на языке Java. Swing был разработан компанией SunMicrosystems. Он содержит ряд графических компонентов, таких как кнопки, поля ввода, таблицы и т. д.

\textbf{Преимущества:} 
\begin{itemize}
	\item Кроссплатформенность;
	\item Компоненты Swing следуют парадигме Model-View-Controller (MVC) и, таким образом, могут обеспечить гораздо более гибкий пользовательский интерфейс;
	\item Swing обеспечивает встроенную двойную буферизацию.
\end{itemize}

\textbf{Недостатки:} 
\begin{itemize}
	\item достаточно узкий спектр возможностей при работе с ui.
	\item считается устаревшей библиотекой.
\end{itemize}


\subsection{JavaFX}

\textbf{JavaFX} -- платформа на основе Java для создания приложений с насыщенным графическим интерфейсом. Может использоваться как для создания настольных приложений, запускаемых непосредственно из-под операционных систем, так и для интернет-приложений, работающих в браузерах, и для приложений на мобильных устройствах. 

JavaFX предназначен для предоставления приложениям таких сложных функций графического интерфейса, как плавная анимация, просмотр веб-страниц, воспроизведение аудио и видео, а также использование CSS стилей.

\textbf{Преимущества:} 
\begin{itemize}
	\item кроссплатформенность;
	\item больше встроенных возможностей.
\end{itemize}

\textbf{Недостатки:} 
\begin{itemize}
	\item в значительной степени зависит от огромной инфраструктуры, которая окружает Java.
\end{itemize}



\subsection*{Вывод}

Уже более 10 лет разработчики приложений считают Swing высокоэффективным инструментарием для создания графических пользовательских интерфейсов (GUI) и добавления интерактивности в Java-приложения. Однако некоторые из самых популярных на сегодняшний день функций графического интерфейса не могут быть легко реализованы с помощью Swing в отличии от JavaFX. 

Также можно писать программы на JavaFX, используя гораздо меньше кода, потому что JavaFX выполняет за нас всю работу. Не нужно регистрировать event listeners, и это делает тело функций более кратким. Кроме того, с помощью механизма привязки JavaFX легко интегрировать компоненты графического интерфейса с базовой моделью. Основываясь на вышесказанном в качестве библиотеки для работы с GUI была выбрана JavaFX.

\section{Описание используемых типов и структур данных}

Для реализации частиц используются следующие типы и структуры данных:
\begin{itemize} 
	\item \textit{VirusParticle} -- класс, хранящий всю информацию о частице.
	\subitem  \textit{Point} -- класс для работы с координатами частицы;
	\subitem \textit{EnvironmentType} -- перечисление сред, в которых может находится частица в данный момент (для отслеживания времени жизни вирусной частицы).
\end{itemize}

\begin{center}
	\captionsetup{justification=raggedright,singlelinecheck=off}
	\begin{lstlisting}[label=lst:sphere,caption=Структуры данных для хранени информации о частицах вируса]
class VirusParticle {
	Point[] arrayName;
	EnvironmentType environmentType;
	int intensity;
}
		
enum EnvironmentType {
	WOOD, PAPER, PLASTIC, AIR
}
		
class Point {
	int x;
	int y;
	int z;
}
	\end{lstlisting}
\end{center}

Для реализации объектов сцены используются следующие типы и структуры данных:

\begin{itemize} 
	\item \textit{Box} -- отображение стен и пола в помещении;
	\item \textit{Person} -- отображение абстрактной фигуры человека;
	\item \textit{Texture} -- обеспечивает загрузку из файла текстуры, ее интерпретацию на простую поверхность;
	\item \textit{Scene} -- характеризует набор объектов и их свойств
\end{itemize}


\section*{Вывод}

В данном разделе были рассмотрены технологии, которые будут использованы при реализации ПО. В качестве языка была выбрана $Java$.  Для реализации пользовательского интерефейса выбрана библиотека $JavaFX$. Среда разработки -  \textit{IntelliJ IDEA}. Также были описаны используемые типы и структуры данных.
