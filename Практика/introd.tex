\chapter*{Введение}
\addcontentsline{toc}{chapter}{Введение}

 С развитием компьютерных технологий компьютерная графика приобрела совершенно новый статус, поэтому сегодня она является основной технологией в цифровой фотографии, кино, видеоиграх, а также во многих специализированных приложениях. Было разработано большое количество алгоритмов отображения. Главными критериями, которые к ним предъявляются, являются реалистичность изображения и скорость отрисовки. Однако зачастую чем выше реалистичность, тем больше времени и памяти требуется для работы алгоритма.

Одним из направлений моделирования является моделирование движения частиц. Имеется огромная потребность в качественной и эффективной отрисовке распространения частиц вируса. Особенно эта тема стала актуальной после начала пандемии короновируса. Пандемия COVID-19 повлияла на жизнь миллионов людей по всему миру. Помимо серьезных последствий для здоровья, пандемия также изменила нашу повседневную жизнь, перевернула рынок вакансий и подорвала экономическую стабильность. В данном курсовом проекте речь пойдет о моделировании распростарнения частиц вирусной инфекции.

Цели данной работы - подготовить всю необходимую базу для реализации программного обеспечения, которое предоставляет возможность моделировать распространение частиц короновирусной инфекции в помещении.

Задачи, которые необходимо выполнить для достижения поставленной цели:
\begin{itemize}
	\item изучить алгоритмы удаления невидимых линий и поверхностей и методы закраски;
	\item проанализировать алгоритмы, моделирующие броуновское движение частиц;
	\item выбрать подходящие для решения поставленной задачи алгоритмы;
	\item выявить основные требования для программного обеспечения;
	\item формализовать модель и описать выбранные типы и структуры данных;
	\item выбрать язык программирования и среду разработки.
\end{itemize}
